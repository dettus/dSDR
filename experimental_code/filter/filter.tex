\documentclass{article}
\usepackage{amssymb}
\begin{document}
\section{Filters}
Source: Daniel Ch. von Grueningen, "Digitale Signalverarbeitung", Carl Hanser Verlag 2014, page 251.\\

Apparently, the filter coefficients can be calculted like so
\begin{eqnarray*}
h_{lowpass}[n]&=& \frac{\Omega_u}{\pi}sinc \left(\frac{n\Omega_u}{pi}\right)\\
h_{highpass}[n]&=&-\frac{\Omega_l}{\pi}sinc \left(\frac{n\Omega_l}{pi}\right) \\
h_{bandpass}[n]&=& \frac{\Omega_u}{\pi}sinc \left(\frac{n\Omega_u}{pi}\right)-\frac{\Omega_l}{\pi}sinc \left(\frac{n\Omega_l}{\pi}\right){pi}\\
h_{bandstop}[n]&=& \frac{\Omega_l}{\pi}sinc \left(\frac{n\Omega_l}{pi}\right)-\frac{\Omega_u}{\pi}sinc \left(\frac{n\Omega_u}{\pi}\right){pi}\\
\end{eqnarray*}
With
\begin{eqnarray*}
h_{lowpass}[0]&=&\frac{\Omega_u}{\pi}\\
h_{highpass}[0]&=&1-\frac{\Omega_u}{\pi}\\
h_{bandpass}[0]&=&\frac{\Omega_u-\Omega_l}{\pi}\\
h_{bandstop}[0]&=&1-\frac{\Omega_u-\Omega_l}{\pi}\\
\end{eqnarray*}
\section{Windows}
\begin{tabular}{l|c}
Hamming&$w_{Hamm}[n]=\left\{\begin{array}{l@{:}l}c\cdot\left( 0.54+0.46\cos\left(\frac{2\pi n}{N+1}\right)\right)&|n|\leqslant N/2\\0&otherwise\\\end{array}\right.$\\\hline  %% }
\end{tabular}
\end{document}

